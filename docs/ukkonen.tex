\documentclass[times, utf8, seminar, numeric]{fer}
\usepackage{booktabs}
\usepackage{url}
\usepackage{listings}
\usepackage{color}

\lstset{
language=C,                % choose the language of the code
basicstyle=\footnotesize,       % the size of the fonts that are used for the code
numbers=left,                   % where to put the line-numbers
numberstyle=\footnotesize,      % the size of the fonts that are used for the line-numbers
stepnumber=1,                   % the step between two line-numbers. If it is 1 each line will be numbered
numbersep=5pt,                  % how far the line-numbers are from the code
backgroundcolor=\color{white},  % choose the background color. You must add \usepackage{color}
showspaces=false,               % show spaces adding particular underscores
showstringspaces=false,         % underline spaces within strings
showtabs=false,                 % show tabs within strings adding particular underscores
frame=tb,           % adds a frame around the code
tabsize=2,          % sets default tabsize to 2 spaces
captionpos=b,           % sets the caption-position to bottom
breaklines=false,        % sets automatic line breaking
breakatwhitespace=false,    % sets if automatic breaks should only happen at whitespace
escapeinside={\%*}{*)}          % if you want to add a comment within your code
}


\begin{document}

\renewcommand{\figurename}{Slika}
\renewcommand{\tablename}{Tablica}
\renewcommand{\lstlistingname}{Programski odsječak}

% TODO: Navedite naslov rada.
\title{Izgradnja sufiksnog stabla - Ukkonenov algoritam}

% TODO: Navedite vaše ime i prezime.
\author{Frane Kurtović i Matija Šantl}

% TODO: Navedite ime i prezime mentora.
\voditelj{doc. dr. sc. Mirjana Domazet-Lošo}

\maketitle

\tableofcontents

\chapter{Uvod}
Sufiksno stablo \engl{suffix tree} je stablasta struktura podataka koja sadrži sve sufikse nekog niza znakova \cite{ukkonen}. Sufiksno stablo na elegantan način rješava problem pretraživanja podniza u nizu, precizinije, podniz je moguće pronaći u vremenu proporcionalnom duljini podniza \cite{mile}. Jedna takva primjena je pronalazak najduljeg zajedničkog podniza dvaju nizova u vremenu porporcionalnom zbroju njihovih duljina \cite{gusfield}.
	
	Sufiksno stablo može biti izgrađeno u vremenu linearno ovisnom o duljini ulaznog niza. Današnji standard je \textit{Ukkonenov} algoritam iz 1995. godine. To je prvi \textit{on-line} algoritam za izgradnju sufiksnog stabla \cite{mile}.
		
\section{Osnovni pojmovi}
	Neka je $T = t_1 t_2 ... t_n$ niz znakova nad abecedom $\Sigma$. Svaki niz znakova $x$ takav da je $T=u x v$ za neke (moguće i prazne) nizove $u$ i $v$ nazivamo podnizom od $T$.

U ovom tekstu korištene su sljedeće oznake:
\begin{itemize}
	\item ulazni niz znakova je označen s $T$,
	\item duljina ulaznog niza znakova $T$ je označena s $n$
	\item \textit{prefiks} niza $T$ je podniz $T[1, i]$ te je označen s $T^i$, $1 \le i \le n$
	\item \textit{sufiks} niza $T$ je podniz $T[i, n]$ te je označen s $T_i$, $1 \le i \le n + 1$
	\item posebno, $T_{n+1}$ je prazan sufiks, $T_{n+1} = \epsilon$
	\item skup svih sufiksa niza znakova $T$ je označen s $\sigma(T)$
	\item stoblo sufiksa \engl{suffix trie} je stoblo koje predstavlja $\sigma(T)$
\end{itemize}

\chapter{Opis algoritma}
	\textit{Ukkonenov} algoritam se može predstaviti kao linearna verzija algoritma za sufiksna stobla koji je kvadratne složenosti. Taj je algoritam opisan u poglavlju 2.1. Jednom transparentnom promjenom, koja je opisana u poglavlju 2.3, dolazimo do algoritma linearne složenosti, poznatijeg pod nazivom \textit{Ukkonenov} algoritam za izgradju sufiksnog stabla u linearnom vremenu.

\section{Izgradnja sufiksnog stobla}
	Formalno, sufiksno se stoblo može definirati kao petorka $STrie(T) = (Q \cup {\perp}, root, F, g, f)$, i možemo reći da je takvo stoblo prošireni deterministički konačni automat koji ima stablolik graf prijelaza koji predstavljaju stoblo $\sigma(T)$ koje je prošireno sa tzv. funkcijom sufiksa \engl{suffix function} $f$ i pomoćnim stanjem $\perp$. Skup $Q$ od stanja $STrie(T)$ se može staviti u jedan-naprema-jedan vezu s podnizovima niza $T$. Stanje koje odgovara podnizu $x$ oznacavamo s $\overline{x}$.
	
	Početno stanje $root$ odgovara praznom nizu $\epsilon$, a skup F završnom skupu stanja  $\sigma(T)$. Funkcija prijelaza $g$ je definirana kao $g(\overline{x}, a) = \overline{y}$ za sve $\overline{x}$, $\overline{y}$ u $Q$ tako da $y = x a$ pri čemu je $a \in \Sigma$.
	
	Funkcija sufiksa $f$ je definirana za svako stanje $\overline{x} \in Q$ na sljedeći način. Neka je $\overline{x} \ne root$. Onda $x = ay$ za neki $a \in \Sigma$, i $f(\overline{x}) = \overline{y}$. Dodatno, $f(root) = \perp$.
	
	Pomoćno stanje $\perp$ nam omogućava pisanje algoritama u nastavcima tako da se izbjegne eksplicitna razlika između praznih i ne-praznim sufiksa. Stanje $\perp$ je spojeno na stoblo s $g(\perp, a) = root$ za svaki $a \in \Sigma$. 
	
	U skladu s \cite{ukkonen}, funkciju sufiksa $f(r)$ nazivamo sufiksnom vezom \engl{suffix link} stanja $r$. 
	
	On-line izgradnja sufiksnog stobla $STrie(T)$ radi se prolaskom niz $T$ s lijeva na desno. Kao međurezultat izgradnje dobijamo $STrie(T^i)$ za $i=0, 1, ..., n$. 
	Ključna primjedba koja opisuje kako se $STrie(T^i)$ dobija iz $STrie(T^{i-1})$ je da se sufiksi od $T^i$ mogu dobiti dodavanjem $t_i$ na kraj svakog sufiksa od $T^{i-1}$ i dodavanjam praznog sufiksa, tj. $$ \sigma(T^i) = \sigma(T^{i-1}) \cup {\epsilon}$$
	
	Prema definiciji, $STrie(T^{i-1})$ prihvaća $\sigma{T^{i-1}}$. Kako bismo dobili da prihvaća i $\sigma{T^{i}}$, moramo proučiti skup završnih stanja $F_{i-1}$ od $STrie(T^{i-1})$. Ako $r \in F_{i-1}$ još nije $t_{i}$-prijelaz , prijelaz od $r$ prema novom stanju se dodaje. Stanja u koja vode neki stari ili novi $t_{i}$-prijelazi iz nekog stanja $F_{i-1}$ zajedno sa stanjem $root$ čine završna stanja $F_i$ od $STrie{T^{i}}$.
	
	Stanja $r \in F_{i-1}$ koja dobe nove prijelaze mogu se pronaći koristeći sufiksne veze na sljedeći način. Definicija sufiksne funkcije implicira da je $r \in F_{i-1}$ ako i samo ako $r = f^{j} (\overline{t_1 ... t_{i-1}}) $ za neki $0 \le j \le i-1$. Zbog toga su sva stanja u $F_{i-1}$ na putu sufiksne veze koja počinje na najdubljem stanju $\overline{t_1 .. t_{i-1}}$ od $STrie(T^{i-1})$ i završava na $\perp$. Taj se put zove granični put \engl{boundary path} od $STrie(T^{i-1})$.
	
	Ako stanje $\overline{z}$ na graničnom putu još nema prijelaz na $t_i$, novo stanje $\overline{z t_i}$ i novi prijelaz $g(\overline{z}, t_i) = \overline{z t_i}$ se dodaje. Time osvježavamo funkciju prijelaza, $g$. Funkciju sufiksa, $f$, osvježavamo tako da nova stanja $\overline{z t_i}$ povezujemo s novim sufiksnim vezama koje čine put koji počinje u stanju $\overline{t_1 t_2 .. t_i}$, tj. granični put od $STrie(T^i)$. 
	
	Prolazak preko $F_{i-1}$ na graničnom putu se može odmah zaustaviti kada je pronađeno prvo stanje $\overline{z}$ takvo da stanje $\overline{z t_i}$ (a time i prijelaz $g(\overline{z}, t_i) = \overline{z t_i}$) već postoji. Neka je $\overline{z t_i}$ stanje. Onda $STrie(T^{i-1})$ mora sadržavati stanje $\overline{z\rq{} t_i}$ i prijelaz $g(\overline{z\rq{}}, t_i) = \overline{z\rq{} t_i}$ za svaki $\overline{z\rq{}} = f^j(\overline{z})$, $j \ge 1$. Drugim riječima, ako je $z t_i$ podniz od $T^{i-1}$ onda je svaki sufiks od $z t_i$ podniz od $T^{i-1}$. Stanje $\overline{z}$ će uvijek postojati jer je stanje $\perp$ zadnje stanje na graničnom putu i stanje $\perp$ ima definiran prijelaz za svaki mogući $t_i$.
	
	Kada je takav obilazak završio, stvorit ćemo novo stanje za svaku sufiks vezu koju smo pregledali tijekom obilaska. To implicira da će cijeli postupak trajati ovisno o veličini dobivenog automata.
	
	U nastavku je prikazan algoritam za izgradnju stobla $STrie{T^i}$ iz $STrie(T^{i-1})$, pri čemu je \textit{top} zapravo stanje $\overline{t_1 ... t_{i-1}}$;
	
\begin{lstlisting}[caption=Izgradnja $STrie(T^i)$ iz $STrie(T^{i-1})$, label=algo1]
r = top;
dok je g(r, t[i]) nedefinirano
	stvori novo stanje r` i novi prijelaz g(r, t[i]) = r`;
	ako je r != top onda: stvori novu sufiksnu vezu f(oldr`) = r`;
	oldr` = r`;
	r = f(r);
stvori novu sufiksnu vezu f(oldr`) = g(r, t`);
top = g(top, t[i])
\end{lstlisting}
	
	Počevši od $STrie(\epsilon)$, koje se sastoji samo od stanja $root$, $\perp$ i veza između njih, ponavljanjem procedure \ref{algo1} za $t_i = t_1,t_2,...t_n$ dobijamo $STrie(T)$. Algoritam je optimalan u smislu da je vremenska složenost proporcionalna konačnom rezultatu, $STrie(T)$, a to je proporcionalno s $|Q|$, tj. brojem različitih podnizova od $T$. Međutim, to može dovesti do kvadratne složenosti u $|T|$, ako je npr. $T=a^n b^n$.

\section{Sufiksno stablo}
	Sufiksno stablo $STree(T)$ nad nizom $T$ je struktura podataka koja predstavlja $STrie(T)$ u prostoru koji je linearan s duljinom $|T|$ niza $T$ \cite{ukkonen}. To se postiže tako da ne koristimo sva, već samo podskup stanja $Q\rq{} \cup {\perp}$ svih stanja od $STrie(T)$. Ta stanja nazivamo eksplictna stanja \engl{explicit states}. Skup $Q\rq{}$ se sastoji od granajućih stanja \engl{branching state}, stanja koja imaju barem dva prijelaza, i listova \engl{leaves}, stanja koja nemaju prijelaza. Po definiciji, stanje $root$ je uključeno u granajuća stanja. Sva preostala stanja $STrie(T)$ zovemo implicitna stanja \engl{implicit states} sufiksnog stabla $STree(T)$.
	
	Niz znakova $w$ koji se nalazi na putu u stoblu $STrie(T)$ između dva eksplicitna stanja $s$ i $r$ je u stablu $STree(T)$ predstavljen kao generalizirani prijelaz $g\rq{}(s, w) = r$. Kako bi uštedjeli na prostoru, niz znakova $w$ je zapravo prikazan uređenim parom pokazivača $(k, p)$ na $T$ takvih da je $w = t_k ... t_p$. Zbog toga, generalizirani prijelaz ima oblik $g\rq{}(s,(k, p)) = r$.
	
	Prijelaz $g\rq{}(s, (k, p)) = r$ nazivamo \textit{a-prijelaz} ako je $t_k = a$. Svako stanje $s$ može imati najviše jedan \textit{a-prijelaz} za svaki $a \in \Sigma$.
	
	Prijelazi $g(\perp, a) = root$ su prikazani na sličan način. Neka je $\Sigma = {a_1, a_2, ..., a_m}$. Onda je prijelaz $g(\perp, a_j) = root$ prikazan kao $g(\perp, (-j, -j)) = root$ za $j = 1, ... , m$.
	
	Zbog toga, sufiksno stablo $STree(T)$ ima dvije komponente: niz znakova $T$ i samo stablo. Ono je linearne veličine u $|T|$ jer skup $Q\rq{}$ ima najviše $|T|$ listova (postoji najviše jedan list za svaki ne-prazni sufiks) i zbog toga skup $Q\rq{}$ mora sadržavati najviše $|T| - 1$ granajućih stanja. Može postojati najviše $2|T| - 2$ prijelaza između stanja u skupu $Q\rq{}$, gdje svaki prijelaz zauzima konstatan prostor zbog korištenja pokazivača.
	
	Stablo se dodatno proširuje sa sufiksnom funkcijom $f\rq{}$, koja je sada definirana za sva granajuća stanja $\overline{x} \ne root$ kao $f\rq{}(\overline{x}) = \overline{y}$ gdje je $y$ takvo granajuće stanje da je $x = ay$ za neki $a \in \Sigma$, i $f\rq{}(root) = \perp$. 
	
	Formalno, sufiksno stablo možemo predstaviti uređenom četvorkom kao $STree(T) = (Q\rq{} \cup {\perp}, root, g\rq{}, f\rq{})$.
	
	Na neko implicitno ili eksplicitno stanje sufiksnog stabla $r$ se referenicramo pomoću uređenog para $s, w$ gdje je $s$ neko eksplicitno stanje koje prethodi stanju $r$ a $w$ je niz znakova koje posjećujemo na putu od $s$ prema $r$ u odgovarajućem sufiksnom stoblu. Referencijski par \engl{reference pair} se naziva kanonskim \engl{canonical} ako je stanje $s$ najbliži predak stanja $r$ (i prema toma niz $w$ je najkraći mogući). Za nako eksplicitno stanje $r$, kanonski referencijski par je $(r, \epsilon)$, tj. budući da koristimo pokazivače, taj par poprima oblik $(r, (p+1, p))$ pri čemu je $w = t_k ... t_p$.
	
	Eksplicitna završrna stanja dobivamo dodavanjem oznake kraja na $T$ koja se ne pojavljuje nigdje u $T$.

\section{On-line izgradnja sufiksnog stabla}

\chapter{Prikaz rada algoritma}


\chapter{Rezultati testiranja}
Rezultati.

\chapter{Zaključak}
Zaključak.

\bibliography{literatura}
\bibliographystyle{fer}

\chapter{Sažetak}
Sažetak.

\end{document}
